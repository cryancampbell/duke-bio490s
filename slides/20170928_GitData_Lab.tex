%%%%%%%%%%%%%%%%%%%%%%%%%%%%%%%%%%%%%%%%%
% Beamer Presentation
% LaTeX Template
% Version 1.0 (10/11/12)
%
% This template has been downloaded from:
% http://www.LaTeXTemplates.com
%
% License:
% CC BY-NC-SA 3.0 (http://creativecommons.org/licenses/by-nc-sa/3.0/)
%
%%%%%%%%%%%%%%%%%%%%%%%%%%%%%%%%%%%%%%%%%

%----------------------------------------------------------------------------------------
%	PACKAGES AND THEMES
%----------------------------------------------------------------------------------------

\documentclass[14pt]{beamer}

\mode<presentation> {

% The Beamer class slide themes
\usetheme{Madrid} %i was using this one

% Beamer class color themes

%\usecolortheme{albatross}

%\setbeamertemplate{footline} % To remove the footer line in all slides uncomment this line
%\setbeamertemplate{footline}[page number] % To replace the footer line in all slides with a simple slide count uncomment this line

%\setbeamertemplate{navigation symbols}{} % To remove the navigation symbols from the bottom of all slides uncomment this line
}

\usepackage{graphicx} % Allows including images
\usepackage{booktabs} % Allows the use of \toprule, \midrule and \bottomrule in tables
\usepackage{hyperref}
\usepackage{helvet}

%----------------------------------------------------------------------------------------
%	TITLE PAGE
%----------------------------------------------------------------------------------------

\title[Git and SRA]{Sharing Scripts \& Data} % The short title appears at the bottom of every slide, the full title is only on the title page

\author{C. Ryan Campbell} % Your name
\institute[Duke] % Your institution as it will appear on the bottom of every slide, may be shorthand to save space
{
Duke University \\ % Your institution for the title page
\medskip
\textit{c.ryan.campbell@duke.edu} % Your email address
}
\date{28 Sept 2017} % Date, can be changed to a custom date

\begin{document}

\begin{frame}
\titlepage % Print the title page as the first slide
\end{frame}

\begin{frame}
\frametitle{Overview} % Table of contents slide, comment this block out to remove it
\tableofcontents % Throughout your presentation, if you choose to use \section{} and \subsection{} commands, these will automatically be printed on this slide as an overview of your presentation
\end{frame}

%----------------------------------------------------------------------------------------
%	PRESENTATION SLIDES
%----------------------------------------------------------------------------------------

%------------------------------------------------
\section{Goals} % A subsection can be created just before a set of slides with a common theme to further break down your presentation into chunks
%------------------------------------------------

%------------------------------------------------
\begin{frame}
\frametitle{Today's Goals}
\begin{itemize}
	\item<+-> Set up a group git
	\begin{itemize}
		\item<+-> Fetch and Commit
		\item<+-> Push and Pull
		\item<+-> Access files in cluster folder 
	\end{itemize}
	\item<+-> Download project data
	\begin{itemize}
		\item<+-> \texttt{wget, curl, sratoolkit}
	\end{itemize}
\end{itemize}
\end{frame}

%------------------------------------------------
\begin{frame}
\frametitle{First, an Assignment}
\begin{itemize}
	\item<+-> Tuesday we'll read and discuss 4 RNAseq/gene expression papers
	\item<+-> First task - pick 4 of these 6
	\begin{enumerate}
		\item<+-> Marsupial/Eutherian Lactation
		\item<+-> Butterfly Wing Pattern
		\item<+-> Sex Difference in Mouse Olfaction
		\item<+-> Hibernation in Dwarf Lemurs
		\item<+-> Alzheimer's Disease and Astrocytes
		\item<+-> Obesity in Chickens
	\end{enumerate}
\end{itemize}
\end{frame}

%------------------------------------------------
\begin{frame}
\frametitle{First, an Assignment}
\begin{itemize}
	\item<+-> These 4 papers to be read by groups of 3 or 4 students
	\item<+-> We'll ``jigsaw'' the papers on Tuesday
	\item<+-> To decide who gets which paper we'll do another guessing game:
\end{itemize}
\end{frame}

%------------------------------------------------
%%\begin{frame}
%%\frametitle{Guessing Game - Phones Down}
%%\begin{itemize}
%%	\item<+-> Who holds the \underline{NCAA career assists} record?
%%	\item<+-> How many assists is it?
%%	\item<+-> Guess the number and write it and your name on a piece of paper
%%	\begin{itemize}
%%		\item<+-> Here's a guide - there are 30-35 games played by a college team in a year
%%		\item<+-> So, use that to do some math...
%%	\end{itemize}
%%	\item<+-> The most accurate guess will get first pick of the papers
%%	\item<+-> Make your guesses now, you have 60 seconds 
%%\end{itemize}
%%\end{frame}

%------------------------------------------------
\section{Gitting}
%------------------------------------------------

%------------------------------------------------
\begin{frame}
\frametitle{GitHub}
\begin{itemize}
	\item<+-> GitHub is a version control and file sharing service
	\item<+-> Works on your laptops as well as the cluster (via terminal)
	\item<+-> We're going to use GitHub for the projects so that:
	\begin{itemize}
		\item<+-> You can all access and edit the same files
		\item<+-> Those files can be transferred easily to the cluster
		\item<+-> You aren't submitting redundant commands
	\end{itemize}
\end{itemize}
\end{frame}

%------------------------------------------------
\begin{frame}
\frametitle{GitHub}
\begin{itemize}
	\item<+-> GitHub is a version control and file sharing service
	\item<+-> Works on your laptops as well as the cluster (via terminal)
	\item<+-> We're going to use GitHub for the projects so that:
	\begin{itemize}
		\item<+-> You can all access and edit the same files
		\item<+-> Those files can be transferred easily to the cluster
		\item<+-> You aren't submitting redundant commands
	\end{itemize}
\end{itemize}
\end{frame}

%------------------------------------------------
\begin{frame}
\frametitle{GitHub}
\begin{itemize}
	\item<+-> This will all take place in the course Git repo
	\item<+-> You'll need to clone it to your computer
	\item<+-> Then give me your git username so I can make you a collaborator
\end{itemize}
\end{frame}

%------------------------------------------------
\begin{frame}
\frametitle{GitHub}
\begin{itemize}
	\item<+-> Go to the GitHub site and copy the ``clone URL''
	\item<+-> https://github.com/cryancampbell/duke-bio490s.git
	\item<+-> Using one of SourceTree/github/terminal clone the repo to your computer
	\item<+-> \texttt{git clone https://github.com/cryancampbell/duke-bio490s.git}
\end{itemize}
\end{frame}

%------------------------------------------------
\subsection{Fetch, Commit, Push \& Pull}
%------------------------------------------------

%------------------------------------------------
\begin{frame}
\frametitle{Fetch, Commit, Push \& Pull}
\begin{itemize}
	\item<+-> Fetch, Commit, Push \& Pull
	\begin{itemize}
		\item<+-> The four main commands to send data back and forth to git
	\end{itemize}
	\item<+-> Again, they can be used in the terminal:
	\item<+-> e.g. - \texttt{git fetch}
	\item<+-> Or the buttons on the SourceTree or github
\end{itemize}
\end{frame}

%------------------------------------------------
\begin{frame}
\frametitle{Fetch, Commit, Push \& Pull}
\begin{itemize}
	\item<+-> Fetch - checks with the repo to see what changes you've missed
	\item<+-> Pull - changes your local machine to reflect those changes
	\item<+-> Commit - prepares local changes to be pushed to the repo
	\item<+-> Push - changes the repo to reflect the changes you've committed
\end{itemize}
\end{frame}

%------------------------------------------------
\begin{frame}
\frametitle{Fetch, Commit, Push \& Pull}
\begin{itemize}
	\item<+-> Fetch - checks with the repo to see what changes you've missed
	\item<+-> Pull - changes your local machine to reflect those changes (aka ``merge'')
	\item<+-> Commit - prepares local changes to be pushed to the repo (it isn't tracking every change you make)
	\item<+-> Push - changes the repo to reflect the changes you've committed
\end{itemize}
\end{frame}

%------------------------------------------------
\begin{frame}
\frametitle{GitHub}
\begin{itemize}
	\item<+-> This folder is for SCRIPTS not raw data
	\item<+-> (Otherwise we'd all need GB of laptop space)
	\item<+-> So remember to send downloaded data \underline{to your own folder on the cluster}
	\item<+-> e.g. - \texttt{/work/cc216/490S/<your folder>} 
\end{itemize}
\end{frame}

%------------------------------------------------
\section{Data}
%------------------------------------------------

%------------------------------------------------
\begin{frame}
\frametitle{Data}
\begin{itemize}
	\item<+-> Many journals and grants require data to be public
	\item<+-> PLOS, NSF, NIH
	\item<+-> The information for where data can be found is often in:
	\begin{itemize}
		\item<+-> The ``Methods'' section of a paper
		\item<+-> An appendix
		\item<+-> The author's lab website
	\end{itemize}
	\item<+-> We'll talk about how to get data from each of these places 
\end{itemize}
\end{frame}

%------------------------------------------------
\subsection{Commandline Tools}
%------------------------------------------------

%------------------------------------------------
\begin{frame}
\frametitle{Downloand Commands}
\begin{itemize}
	\item<+-> \texttt{curl} 
	\item<+-> \texttt{wget}
	\item<+-> \texttt{fastq-dump} - subset of the SRA Toolkit
	\item<+-> Each is useful in different situations
\end{itemize}
\end{frame}

%------------------------------------------------
\begin{frame}
\frametitle{curl}
\begin{itemize}
	\item Downloads a given url site
	\ttfamily
	\begin{block}{}
		\item[] Usage: curl [options...] <url>
	\end{block}
	\sffamily
	\item Options include username \& password
	\item Useful for sftp servers
	\item An example:
	\ttfamily
	\footnotesize
	\begin{block}{}
		\item[] curl -o ESPNfrontpage.html http://www.espn.com/
	\end{block}
	\item \textsf{saved as} ESPNfrontpage.html
\end{itemize}
\end{frame}

%------------------------------------------------
\begin{frame}
\frametitle{wget}
\begin{itemize}
	\item Downloads a given url site
	\ttfamily
	\begin{block}{}
		\item[] Usage: wget [OPTION]... [URL]...
	\end{block}
	\sffamily
	\item Same options as curl
	\item An example:
	\ttfamily
	\footnotesize
	\begin{block}{}
		\item[] wget http://www.bzip.org/1.0.6/bzip2-1.0.6.tar.gz
	\end{block}
	\normalsize
	\sffamily
	\item Often used for downloading software (you'll see later...)
\end{itemize}
\end{frame}

%------------------------------------------------
\subsection{SRA}
%------------------------------------------------

%------------------------------------------------
\begin{frame}
\frametitle{Sequence Read Archive}
\begin{itemize}
	\item<+-> Holds raw sequence data from published articles
	\item<+-> Sorted by Experiment and Project
	\item<+-> Use SRA Toolkit to access the files
\end{itemize}
\end{frame}

%------------------------------------------------
\begin{frame}
\frametitle{Sequence Read Archive}
\begin{itemize}
	\item<+-> Often directly reference in papers
	\item<+-> e.g. ``data are available in SRR\#\#\#\#\#\#\#\#''
	\item<+-> Sometimes they are well organized, sometimes not
	\item<+-> You should be able to follow the paper's methodology to separate the data into samples (if it wasn't kept that way on SRA)
\end{itemize}
\end{frame}

%%%%UNCOMMMENT
%------------------------------------------------
\begin{frame}
\frametitle{SRA Toolkit}
\begin{itemize}
	\item<+-> So, how do we download the data?
	%%\item<+-> Google "SRA Toolkit" or click download link on main page
	%%\item<+-> Which version do we want?
	%%\item<+-> Duke's cluster is CentOS Linux 64 bit
\end{itemize}
\end{frame}

%------------------------------------------------
\begin{frame}
\frametitle{SRA Toolkit}
\begin{itemize}
	\item<+-> How do we get the software onto the cluster?
	%%\item<+-> \texttt{wget}
	%%\item<+-> What is the full command?
	%%\item<+-> \texttt{wget https://ftp-trace.ncbi.nlm.nih.gov/sra/sdk/2.8.2-1/sratoolkit.2.8.2-1-centos_linux64.tar.gz}
	%%\item<+-> Where should we put it? 
\end{itemize}
\end{frame}

%------------------------------------------------
\begin{frame}
\frametitle{SRA Toolkit}
\begin{itemize}
	\item<+-> How do we use this file?
	%%\item<+-> google ".tar.gz" extension
	%%\item<+-> What is a tar file?
	%%\item<+-> A group of files and folders packaged together
	%%\item<+-> What is the command? 
	%%\item<+-> \texttt{tar xfvz sratoolkit.current-centos_linux64.tar.gz}
	%%\item<+-> Now what?
\end{itemize}
\end{frame}

%------------------------------------------------
\begin{frame}
\frametitle{SRA Toolkit}
\begin{itemize}
	\item How do we use the toolkit?
	%%\item[] Sometimes the README is helpful 
	\item What is the toolkit (file, command, etc)?
	%%\item Run it and see what happens!
	\ttfamily
	\footnotesize
	%%\begin{block}{}
	%%\item[] /<your path>/sratoolkit.2.8.2-1-centos\_linux64/bin/fastq-dump
	%%\item[] /<your path>/sratoolkit.2.8.2-1-centos\_linux64/bin/fastq-dump --help
	%%\end{block}
\end{itemize}
\end{frame}

%%%%UNCOMMENT
%------------------------------------------------
%%\begin{frame}
%%\frametitle{SRA Toolkit}
%%\begin{itemize}
%%	\item I've placed an example file on the cluster:
%%	\ttfamily
%%	\footnotesize
%%	\begin{block}{}
%%	\item[] cat /work/cc216/490S/duke-bio490s/projects/sra\_dl\_EXAMPLE.submit
%%	\end{block}
%%\end{itemize}
%%\end{frame}

%------------------------------------------------
\begin{frame}
\frametitle{Group Work}
\begin{itemize}
	\item[] Group steps
	\begin{enumerate}
		\item Pick a group ID/Name
		\item Make a group folder with that name within:
		\item \texttt{duke-bio490s/projects/<group name here>}
	\end{enumerate}
	\item This is where you'll keep scripts to download, trim, etc data
\end{itemize}
\end{frame}

%------------------------------------------------
\begin{frame}
\frametitle{Group Work}
\begin{itemize}
	\item<+-> You should have everything you need now to download your own data
	\item<+-> TIPS:
	\begin{itemize}
		\item<+-> Run the line of code in the terminal \underline{FIRST} so you can troubleshoot it
		\item<+-> Instead of downloading the whole file use \texttt{head} to check output  
		\item<+-> Once your output looks correct \underline{THEN} submit to the cluster
		\item<+-> This will save a lot of headaches and waiting
		\item<+-> And most importantly: \underline{DON'T write data to the git repo}
	\end{itemize}
\end{itemize}
\end{frame}

%------------------------------------------------
\begin{frame}
\Huge{\centerline{The End}}
\end{frame}

%----------------------------------------------------------------------------------------

\end{document} 