%%%%%%%%%%%%%%%%%%%%%%%%%%%%%%%%%%%%%%%%%
% Beamer Presentation
% LaTeX Template
% Version 1.0 (10/11/12)
%
% This template has been downloaded from:
% http://www.LaTeXTemplates.com
%
% License:
% CC BY-NC-SA 3.0 (http://creativecommons.org/licenses/by-nc-sa/3.0/)
%
%%%%%%%%%%%%%%%%%%%%%%%%%%%%%%%%%%%%%%%%%

%----------------------------------------------------------------------------------------
%	PACKAGES AND THEMES
%----------------------------------------------------------------------------------------

\documentclass[14pt]{beamer}

\mode<presentation> {

% The Beamer class slide themes
\usetheme{Madrid} %i was using this one

% Beamer class color themes

%\usecolortheme{albatross}

%\setbeamertemplate{footline} % To remove the footer line in all slides uncomment this line
%\setbeamertemplate{footline}[page number] % To replace the footer line in all slides with a simple slide count uncomment this line

%\setbeamertemplate{navigation symbols}{} % To remove the navigation symbols from the bottom of all slides uncomment this line
}

\usepackage{graphicx} % Allows including images
\usepackage{booktabs} % Allows the use of \toprule, \midrule and \bottomrule in tables
\usepackage{hyperref}
\usepackage{helvet}

%----------------------------------------------------------------------------------------
%	TITLE PAGE
%----------------------------------------------------------------------------------------

\title[bash \& SLURM]{Introduction to bash \& SLURM} % The short title appears at the bottom of every slide, the full title is only on the title page

\author{C. Ryan Campbell} % Your name
\institute[Duke] % Your institution as it will appear on the bottom of every slide, may be shorthand to save space
{
Duke University \\ % Your institution for the title page
\medskip
\textit{c.ryan.campbell@duke.edu} % Your email address
}
\date{14 Sept 2017} % Date, can be changed to a custom date

\begin{document}

\begin{frame}
\titlepage % Print the title page as the first slide
\end{frame}

\begin{frame}
\frametitle{Overview} % Table of contents slide, comment this block out to remove it
\tableofcontents % Throughout your presentation, if you choose to use \section{} and \subsection{} commands, these will automatically be printed on this slide as an overview of your presentation
\end{frame}

%----------------------------------------------------------------------------------------
%	PRESENTATION SLIDES
%----------------------------------------------------------------------------------------

%------------------------------------------------
\section{Goals} % A subsection can be created just before a set of slides with a common theme to further break down your presentation into chunks
%------------------------------------------------

%------------------------------------------------
\begin{frame}
\frametitle{Today's Goals}
\begin{itemize}
	\item<+-> Log into the cluster
	\item<+-> Learn some basic commands
	\begin{itemize}
		\item<+-> cd, mkdir, ls, scp
		\item<+-> sort, uniq, cut, nano
	\end{itemize}
	\item<+-> Learn some advanced commands
	\begin{itemize}
		\item<+-> grep, sed
	\end{itemize}
	\item<+-> Submit a cluster job
	\item<+-> See how far we get
\end{itemize}
\end{frame}

%------------------------------------------------
\section{Cluster Basics}
%------------------------------------------------

%------------------------------------------------
\subsection{Logging On}
%------------------------------------------------

%------------------------------------------------
\begin{frame}
\frametitle{Logging On \& Aliases}
\begin{itemize}
	\item Use command ``ssh''
	\item Logs on to one of two compute nodes:
	\ttfamily
	\begin{block}{}
		\item[] ssh <netid>@dscr-slogin-01.oit.duke.edu
		\item[] ssh <netid>@dscr-slogin-02.oit.duke.edu
	\end{block}
	\sffamily
	\item Your terminal window is now on the computer ``dscr-slogin-01.oit.duke.edu''
	\item Note the command prompt
\end{itemize}
\end{frame}

%------------------------------------------------
\begin{frame}
\frametitle{Logging On \& Aliases}
\begin{itemize}
	\item<+-> To log in from off campus use a VPN
	\item vpn.duke.edu
	\item[]
	\item<+-> Once you install the software you can access the slogin computer from off campus
\end{itemize}
\end{frame}

%------------------------------------------------
\begin{frame}
\frametitle{Logging On \& Aliases}
\begin{itemize}
	\item Aliases are used to save common commands
	\item On my computer the ssh login is saved as ``DSCR''
	\ttfamily
	\footnotesize
	\begin{block}{}
		\item[] alias DSCR="ssh cc216@dscr-slogin-01.oit.duke.edu"
	\end{block}
	\sffamily
	\normalsize
	\item To set up an alias on your local machine edit your bash profile
	\ttfamily
	\begin{block}{}
		\item[] nano \~/.bash\_profile
	\end{block}
\end{itemize}
\end{frame}

%------------------------------------------------
\begin{frame}
\frametitle{nano}
\begin{itemize}
	\item nano is a text-based text editor
	\item<+-> Just running ``\texttt{nano}'' opens an unsaved text document
	\item<+-> Adding a filename ``\texttt{nano <file>}'' opens that file
	\item[]
	\item<+-> ctrl+O saves the file
	\item (O for ``write Out'')
	\item ctrl+X exits
	\item (X for ``eXit'')
\end{itemize}
\end{frame}

%------------------------------------------------
\begin{frame}
\frametitle{Logging On \& Aliases}
\begin{itemize}
	\item Now that you know more about nano
	\item On you local computer run:
	\ttfamily
	\begin{block}{}
		\item[] nano \~/.bash\_profile
	\end{block}
	\sffamily
	\item Then copy and paste this into the file
	\ttfamily
	\footnotesize
	\begin{block}{}
		\item[] alias DSCR="ssh <netid>@dscr-slogin-01.oit.duke.edu"
	\end{block}
	\sffamily
	\normalsize
	\item And save the file (ctrl+O)
	\item Once you close and re-open the terminal the alias will work
\end{itemize}
\end{frame}

%------------------------------------------------
\subsection{Getting Around}
%------------------------------------------------

%------------------------------------------------
\begin{frame}
\frametitle{Getting Around}
\begin{itemize}
	\item You're on the cluster, now what?
	\item Where are you?
	\ttfamily
	\begin{block}{}
		\item[] pwd
	\end{block}
	\sffamily
	\item[] \texttt{pwd} = print working directory 
	\item Change into our course directory
	\item Absolute vs. Relative location
	\ttfamily
	\begin{block}{}
		\item[] cd /work/cc216/490S/
	\end{block}
	\sffamily
	\normalsize
	\item[] \texttt{cd} = change directory
	\item Is this a relative or absolute address?
	\item How can you tell?
\end{itemize}
\end{frame}

%------------------------------------------------
\begin{frame}
\frametitle{Looking Around}
\begin{itemize}
	\item You're in our directory, now what?
	\item What's in here?
	\ttfamily
	\begin{block}{}
		\item[] ls
	\end{block}
	\sffamily
	\item[] \texttt{ls} = list
	\item Also works with an absolute address
	\ttfamily
	\begin{block}{}
		\item[] ls /work/cc216/490S/
	\end{block}
	\sffamily
	\normalsize
	\item[] \texttt{ls -l} = list with long format
	\item -l is a ``flag''
\end{itemize}
\end{frame}

%------------------------------------------------
\begin{frame}
\frametitle{Flags and Help}
\begin{itemize}
	\item -l is a ``flag''
	\item General grammar rules:
	\item[] \texttt{<command> <flags> <input>}
	\item Spaces and order matter
	\ttfamily
	\begin{block}{}
		\item[] ls -l /work/cc216/490S
	\end{block}
	\sffamily
	\item If you're unsure what the flags are ``-h'' or ``--help'' often works
	\ttfamily
	\begin{block}{}
		\item[] ls -h
		\item[] ls --help
	\end{block}
\end{itemize}
\end{frame}

%------------------------------------------------
\subsection{File Modification}
%------------------------------------------------

%------------------------------------------------
\begin{frame}
\frametitle{Changing Files}
\begin{itemize}
	\item How do you add a folder?
	\ttfamily
	\begin{block}{}
		\item[] mkdir <name of directory>
		\item[] mkdir /work/cc216/490S/test\_directory
	\end{block}
	\sffamily
	\item[] \texttt{mkdir} = make directory
	\item Try making a directory:
	\ttfamily
	\begin{block}{}
		\item[] mkdir <your netid>
	\end{block}
	\sffamily
	\normalsize
	\item[] \texttt{ls} list to confirm it worked
\end{itemize}
\end{frame}

%------------------------------------------------
\begin{frame}
\frametitle{Changing Files}
\begin{itemize}
	\item There are a couple of options for file and folder manipulation
	\item To create a file:
	\ttfamily
	\begin{block}{}
		\item[] touch <name of file>
	\end{block}
	\sffamily
	\item To copy a file:
	\ttfamily
	\begin{block}{}
		\item[] cp <name of file>
	\end{block}
	\sffamily
	\item To move a file:
	\ttfamily
	\begin{block}{}
		\item[] mv <current name of file> <new name of file>
	\end{block}
	\sffamily
	\item Try these inside of your recently created directory
\end{itemize}
\end{frame}

%------------------------------------------------
\subsection{Submitting Jobs}
%------------------------------------------------

%------------------------------------------------
\begin{frame}
\frametitle{Running Jobs on the Cluster}
\begin{itemize}
	\item The main reason to use the cluster is for additional computing ability
	\item This means either longer runs or higher computing power
	\item To take advantage of this you have to submit a ``job''
	\item The cluster will then place your job in a queue and run it when the time comes
	\item The language of these jobs is bash
\end{itemize}
\end{frame}

%------------------------------------------------
\begin{frame}
\frametitle{Running Jobs on the Cluster}
\begin{itemize}
	\ttfamily
	\scriptsize
	\begin{block}{}
		\item[] \#!/bin/bash
		\item[] \#
		\item[] \#SBATCH --job-name=L2\_L001\_bwa
		\item[] \#SBATCH --output=/work/cc216/microcebus\_gatk/align/L2\_L001.align.out
		\item[] \#SBATCH --error=/work/cc216/microcebus\_gatk/align/L2\_L001.align.err
		\item[] \#
		\item[] \#SBATCH -p yoderlab
		\item[] \#SBATCH --mem=16G
		\item[] \#SBATCH --nodes=1
		\item[] 
		\item[] cd /work/cc216/microcebus\_gatk/align/
		\item[] bwa mem -t 4 -M  mmur3 /work/cc216/microcebus\_gatk/input\_fastq/L2/L2\_L001.r1.trimm.5.20.fastq.gz /work/cc216/microcebus\_gatk/input\_fastq/L2/L2\_L001.r2.trimm.5.20.fastq.gz > L2\_L001.sam
		\item[] samtools view -bt ../chr\_group\_contigs/mouse\_lemur\_sex\_chr.fa L2\_L001.sam > L2\_L001.bam
	\end{block}
\end{itemize}
\end{frame}

%------------------------------------------------
\begin{frame}
\frametitle{Running Jobs on the Cluster}
\begin{itemize}
	\item A simple one to try out:
	\ttfamily
	\scriptsize
	\begin{block}{}
		\item[] \#!/bin/bash
		\item[] \#
		\item[] \#SBATCH --job-name=test\_script
		\item[] \#SBATCH --output=/work/cc216/490S/<your netid>/test.out
		\item[] \#SBATCH --error=/work/cc216/490S/<your netid>/test.err
		\item[] \#
		\item[] \#SBATCH --mem=2G
		\item[] \#SBATCH --nodes=1
		\item[] 
		\item[] cd /work/cc216/490S/<your netid>/
		\item[] date
		\item[] sleep 60
		\item[] date
		\item[] touch script\_is\_done.finish
	\end{block}
\end{itemize}
\end{frame}

%------------------------------------------------
\begin{frame}
\frametitle{Making a Cluster Script}
\begin{itemize}
	\item<+-> What SHOULDN'T you use?
	\item<+-> What SHOULD you use?
	\begin{itemize}
		\item<+-> nano on the cluster
		\item<+-> TextWrangler/Sublime/etc locally
	\end{itemize}
	\item<+-> but how do you get it to the cluster?
\end{itemize}
\end{frame}

%------------------------------------------------
\begin{frame}
\frametitle{Secure Copy}
\begin{itemize}
	\item Used to copy a file down from the cluster or up to the cluster
	\ttfamily
	\scriptsize
	\begin{block}{}
		\item[] scp <name of file to copy> <file location desired>
		\item[] scp <name of file to copy> <location and name if renaming>
	\end{block}
	\sffamily
	\normalsize
	\item To copy a file from a different computer, the location has to contain the computer name:
	\ttfamily
	\scriptsize
	\begin{block}{}
		\item[] scp -v cc216@dcc-slogin-02.oit.duke.edu:/dscrhome/cc216/490S/hsap\_hypoxia\_gene.diff 
		\item[] ~/Documents/git\_repos/duke-bio490s/labs/20170831\_lab\_intro/
	\end{block}
	\normalsize
	\sffamily
	\item use ``slogin-02'' for file transfers
	\item Also useful - cyberduck
\end{itemize}
\end{frame}

%------------------------------------------------
\begin{frame}
\frametitle{Making a Cluster Script}
\begin{itemize}
	\item Now, make the file on your computer or on the cluster
	\item Then use the \texttt{cat} command to check the contents
	\item[] \texttt{cat} = concatenate, writes the contents of a file to the screen
\end{itemize}
\end{frame}

%------------------------------------------------
\begin{frame}
\frametitle{Running a Cluster Script}
\begin{itemize}
	\item Once you have a script ready to run, (doublechecked with \texttt{cat}) we'll use \texttt{sbatch} to submit the script
	\ttfamily
	\begin{block}{}
		\item[] sbatch <name of script>
	\end{block}
	\sffamily
	\normalsize
	\item Then to track the progress, use \texttt{squeue}:
	\ttfamily
	\begin{block}{}
		\item[] squeue -u <your netid> 
	\end{block}
	\normalsize
	\sffamily
	\item This returns the jobs you have in the queue
	\item What do you think happens if you run just \texttt{squeue}?
\end{itemize}
\end{frame}

%------------------------------------------------
\begin{frame}
\frametitle{You Ran a Cluster Script!}
\begin{itemize}
	\item<+-> Ta-Da!! Now you can run a job on the cluster!
	\item<+-> You now have the minimum knowledge to get onto the cluster and use the resource to run software that is unsuitable for your laptop
	\item<+-> Now, there are a lot more things that the cluster and bash can do, mostly involving data manipulation
	\item<+-> Let's delve into some of those!
\end{itemize}
\end{frame}

%------------------------------------------------
\subsection{Basic Tools}
%------------------------------------------------

%------------------------------------------------
\begin{frame}
\frametitle{Tools}
\begin{itemize}
	\item Input and output from commandline
	\begin{itemize}
		\item Using \texttt{>} and \texttt{|}
	\end{itemize}
	\sffamily
	\item \texttt{sort} - sorting
	\item \texttt{uniq} - sort and eliminate duplicates
	\item \texttt{cut} - split a file into columns
\end{itemize}
\end{frame}

%------------------------------------------------
\begin{frame}
\frametitle{Input and Output}
\begin{itemize}
	\item Unless otherwise noted:
	\begin{itemize}
		\item Input comes after the command
		\item Output ``prints to screen''
	\end{itemize}
	\item Using \texttt{>} and \texttt{|} changes that
	\item \texttt{>} takes output and (over)writes to a file
	\ttfamily
	\begin{block}{}
		\item[] ls > list.txt
	\end{block}
	\sffamily
	\normalsize
	\item If you use double \texttt{>>} it appends to a file
	\ttfamily
	\begin{block}{}
		\item[] ls -lt >> list.txt
	\end{block}
\end{itemize}
\end{frame}

%------------------------------------------------
\begin{frame}
\frametitle{Input and Output}
\begin{itemize}
	\item Using \texttt{|} changes input
	\item \texttt{|} takes output and passes it to the next command
	\ttfamily
	\begin{block}{}
		\item[] ls | head -n1
	\end{block}
	\sffamily
	\normalsize
	\item You can string together many \texttt{|}'s to perform complicated actions
\end{itemize}
\end{frame}

%------------------------------------------------
\begin{frame}
\frametitle{sort}
\begin{itemize}
	\item \texttt{sort} takes input and sorts it! (simple, right?)
	\ttfamily
	\begin{block}{}
		\item[] sort <name of file>
		\item[] cat <name of file> | sort 
	\end{block}
	\sffamily
	\normalsize
	\item Common flags:
	\begin{itemize}
		\item -n sorts numerically
		\item -u sorts and only presents unique hits
		\item -r sorts reverse
	\end{itemize}
\end{itemize}
\end{frame}

%------------------------------------------------
\begin{frame}
\frametitle{uniq}
\begin{itemize}
	\item \texttt{uniq} takes input and removes adjacent duplicates! (still simple, right?)
	\ttfamily
	\begin{block}{}
		\item[] uniq <name of file>
		\item[] cat <name of file> | uniq 
	\end{block}
	\sffamily
	\normalsize
	\item Common flags:
	\begin{itemize}
		\item -c counts each unique line
		\item -d reverses the meaning (prints only duplicates)
	\end{itemize}
\end{itemize}
\end{frame}

%------------------------------------------------
\begin{frame}
\frametitle{cut}
\begin{itemize}
	\item \texttt{cut} takes input and divides it into ``columns''
	\ttfamily
	\begin{block}{}
		\item[] cut -d<what to divide by> -f<which columns you want> <name of file>
	\end{block}
	\sffamily
	\normalsize
	\item Common flags:
	\begin{itemize}
		\item -d what to divide by (``,'' `` '' ``tab'')
		\item -c take n characters (-c1-10 takes first 10 characters in each line)
		\item -f which columns you want:
		\begin{itemize}
			\item -f1, first only
			\item -f1-5 one through five
			\item -f1,5 first and fifth only
		\end{itemize}
	\end{itemize}
\end{itemize}
\end{frame}

%------------------------------------------------
\subsection{Advanced Tools}
%------------------------------------------------

%------------------------------------------------
\begin{frame}
\frametitle{Advanced Tools}
\begin{itemize}
	\item The following tools are more advanced and complicated
	\item You will often see them in online forums
	\item You don't have to be a wizard
	\item It is good to be familiar with them
	\scriptsize
	\item[] (Once you are, there isn't a dataset you won't be able to manage)
	\normalsize
	\item \texttt{grep} - regular expression search
	\item \texttt{sed} - search and replace patterns
	\item \texttt{awk} - counting as well as search
\end{itemize}
\end{frame}

%------------------------------------------------

\begin{frame}
\Huge{\centerline{The End}}
\end{frame}

%----------------------------------------------------------------------------------------

\end{document} 