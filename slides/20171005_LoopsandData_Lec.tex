%%%%%%%%%%%%%%%%%%%%%%%%%%%%%%%%%%%%%%%%%
% Beamer Presentation
% LaTeX Template
% Version 1.0 (10/11/12)
%
% This template has been downloaded from:
% http://www.LaTeXTemplates.com
%
% License:
% CC BY-NC-SA 3.0 (http://creativecommons.org/licenses/by-nc-sa/3.0/)
%
%%%%%%%%%%%%%%%%%%%%%%%%%%%%%%%%%%%%%%%%%

%----------------------------------------------------------------------------------------
%	PACKAGES AND THEMES
%----------------------------------------------------------------------------------------

\documentclass[14pt,handout]{beamer}
%%\documentclass[14pt]{beamer}

\mode<presentation> {

% The Beamer class slide themes
\usetheme{Madrid} %i was using this one

% Beamer class color themes

%\usecolortheme{albatross}

%\setbeamertemplate{footline} % To remove the footer line in all slides uncomment this line
%\setbeamertemplate{footline}[page number] % To replace the footer line in all slides with a simple slide count uncomment this line

%\setbeamertemplate{navigation symbols}{} % To remove the navigation symbols from the bottom of all slides uncomment this line
}

\usepackage{graphicx} % Allows including images
\usepackage{booktabs} % Allows the use of \toprule, \midrule and \bottomrule in tables
\usepackage{hyperref}
\usepackage{helvet}
\usepackage[T1]{fontenc}
\usepackage{textcomp}

%----------------------------------------------------------------------------------------
%	TITLE PAGE
%----------------------------------------------------------------------------------------

\title[Loops]{Loops and Conditionals} % The short title appears at the bottom of every slide, the full title is only on the title page

\author{C. Ryan Campbell} % Your name
\institute[Duke] % Your institution as it will appear on the bottom of every slide, may be shorthand to save space
{
Duke University \\ % Your institution for the title page
\medskip
\textit{c.ryan.campbell@duke.edu} % Your email address
}
\date{05 Oct 2017} % Date, can be changed to a custom date

\begin{document}

\begin{frame}
\titlepage % Print the title page as the first slide
\end{frame}

\begin{frame}
\frametitle{Overview} % Table of contents slide, comment this block out to remove it
\tableofcontents % Throughout your presentation, if you choose to use \section{} and \subsection{} commands, these will automatically be printed on this slide as an overview of your presentation
\end{frame}

%----------------------------------------------------------------------------------------
%	PRESENTATION SLIDES
%----------------------------------------------------------------------------------------

%------------------------------------------------
\section{Goals} % A subsection can be created just before a set of slides with a common theme to further break down your presentation into chunks
%------------------------------------------------

%------------------------------------------------
\begin{frame}
\frametitle{Today's Goals}
\begin{itemize}
	\item<+-> Fix group git permissions
	\item<+-> Learn loops
	\item<+-> Download multiple datafiles
\end{itemize}
\end{frame}

%------------------------------------------------
\section{Cluster git repo}
%------------------------------------------------

%------------------------------------------------
\begin{frame}
\frametitle{Cluster repo}
\begin{itemize}
	\item<+-> git repo is on the cluster
	\item<+-> There are some permissions issues
	\item<+-> Whenever anyone adds a file, the default is that no one else can edit it
	\item<+-> Problem - updating the git repo fails!
	\item<+-> Solution - \texttt{chmod}
\end{itemize}
\end{frame}

%------------------------------------------------
\subsection{Permissions}
%------------------------------------------------

%------------------------------------------------
\begin{frame}
\frametitle{Permissions}
\begin{itemize}
	\normalsize
	\item What are permissions?
	\item The users who are given the ability to edit your files
	\item When I run \texttt{ls -l}
	\footnotesize
	\item \texttt{drwxr-xr-x.  2 cc216 root   58 Sep 28 15:43 software}
	\item \texttt{drwxrwxrwx. 21 cc216 root  561 Sep 29 11:49 490S}
	\normalsize
	\item The 9 letters show:
	\item<2-> Read/Write/Execute permissions for User/Group/World
\end{itemize}
\end{frame}

%------------------------------------------------
\begin{frame}
\frametitle{Permissions}
\begin{itemize}
	\item To edit the permissions you use \texttt{chmod}:
	\ttfamily
	\footnotesize
	\begin{block}{}
	\item[] chmod -R 777 /work/cc216/490s/duke-bio490s
	\end{block}
	\normalsize
	\sffamily
	\item What does this do?
	\item If you're not sure how would you find out?
\end{itemize}
\end{frame}

%------------------------------------------------
\subsection{Cron Jobs}
%------------------------------------------------

%------------------------------------------------
\begin{frame}
\frametitle{Cron Jobs}
\begin{itemize}
	\item It would be hard (and annoying) to rememember to run that command as often as you create files
	\item So we'll set up a ``cron job''
	\item This is a command that the computer (slogin) will run repeatedly at a set time
	\footnotesize
	\item To do so run:
	\ttfamily
	\begin{block}{}
	\item[] export VISUAL=nano; crontab -e
	\end{block}
	\sffamily
	\item This should open a blank file with nano	
	\item Inside add the following text (paying attention to spaces):
	\ttfamily
	\begin{block}{}
	\item[] 01 * * * * chmod -R 777 /work/cc216/490S/duke-bio490s > /work/cc216/490S/<your netid>/cron.out 2>\&1
	\end{block}
	\sffamily
	\item Finally save (ctrl + O) and exit (ctrl + X)
\end{itemize}
\end{frame}

%------------------------------------------------
\section{Variables}
%------------------------------------------------

%------------------------------------------------
\begin{frame}
\frametitle{Variables}
\begin{itemize}
	\item Variables store data
	\item No data types
	\item Can be a number, character, or string of characters
	\item Keep names simple, letters only
	\footnotesize
	\ttfamily
	\begin{block}{}
		\item[] STR=\textquotedbl Hello World\textquotedbl
		\item[] echo \$STR
	\end{block}
	\sffamily
	\item[] Use the \$ to call a variable
	\item[] So the computer reads this as: \texttt{echo Hello World}
\end{itemize}
\end{frame}

%------------------------------------------------
\section{Loops}
%------------------------------------------------

%------------------------------------------------
\begin{frame}
\frametitle{Loops}
\begin{itemize}
	\item<+-> Variables are integral to loops
	\item<+-> A loop performs a process interatively as a variable changes
	\item<+-> \texttt{for} - \underline{for} all the items in a list, execute the script
	\item<+-> \texttt{while} - \underline{while} a condition is met, execute the script
	\item<+-> \texttt{until} - \underline{until} a condition is met, execute the script
	\item<+-> while and until are opposites
\end{itemize}
\end{frame}

%------------------------------------------------
\subsection{for}
%------------------------------------------------

%------------------------------------------------
\begin{frame}
\frametitle{for Loop}
\begin{itemize}
	\footnotesize
	\ttfamily
	\begin{block}{}
        \item[] for i in \$( ls ); do
        \item[]    echo item: \$i
        \item[] done
	\end{block}	
	\begin{block}{}
        \item[] for i in \$( ls ); do echo item: \$i; done
	\end{block}
	\item[] OR
	\begin{block}{}
        \item[] for i in `seq 1 10'; 
        \item[] do
        \item[]    echo \$i
        \item[] done
	\end{block}	
	\begin{block}{}
        \item[] for i in `seq 1 10'; do echo \$i; done
	\end{block}
	\sffamily
	\item[] (note that this example isn't a \textquotesingle, it is an angled quote which shares the tilde key)
\end{itemize}
\end{frame}

%------------------------------------------------
\subsection{while/until}
%------------------------------------------------

%------------------------------------------------
\begin{frame}
\frametitle{while Loop}
\begin{itemize}
	\footnotesize
	\ttfamily
	\begin{block}{}
        \item[] COUNTER=0
        \item[] while [  \$COUNTER -lt 10 ]; do
        \item[]     echo The counter is \$COUNTER
        \item[]     let COUNTER=COUNTER+1 
        \item[] done
	\end{block}	
	\begin{block}{}
        \item[] COUNTER=0; while [  \$COUNTER -lt 10 ]; do echo The counter is \$COUNTER; let COUNTER=COUNTER+1; done
	\end{block}
	\sffamily
	\item[] ``let'' allows us to do math with variables, as does \$(( a + b ))
\end{itemize}
\end{frame}

%------------------------------------------------
\begin{frame}
\frametitle{until Loop}
\begin{itemize}
	\item How would we write an ``until'' loop that does the same thing?
	\footnotesize
	\ttfamily
	\begin{block}{}
        \item[] COUNTER=0
        \item[] while [  \$COUNTER -lt 10 ]; do
        \item[]     echo The counter is \$COUNTER
        \item[]     let COUNTER=COUNTER+1 
        \item[] done
	\end{block}	
	\sffamily
	\tiny
\end{itemize}
\end{frame}

%------------------------------------------------
\begin{frame}
\frametitle{until Loop}
\begin{itemize}
	\item ``until'' loop that has the same output:
	\footnotesize
	\ttfamily
	\begin{block}{}
        \item[] COUNTER=0
        \item[] until [  \$COUNTER -gt 9 ]; do
        \item[]     echo The counter is \$COUNTER
        \item[]     let COUNTER=COUNTER+1 
        \item[] done
	\end{block}	
	\begin{block}{}
        \item[] COUNTER=0; until [  \$COUNTER -gt 9 ]; do echo The counter is \$COUNTER; let COUNTER=COUNTER+1; done
	\end{block}
	\sffamily
	\tiny
\end{itemize}
\end{frame}

%------------------------------------------------
\section{Conditionals}
%------------------------------------------------

%------------------------------------------------
\begin{frame}
\frametitle{Conditionals}
\begin{itemize}
	\item<+-> Logical parameters
	\item<+-> Are powerful when automating processes
	\item<+-> \texttt{if} - \underline{if} a condition is met (evaluates T/F)
	\item<+-> \texttt{then} - \underline{then} execute the script
	\item<+-> \texttt{else} - \underline{until} if the condition is not met, execute this script
\end{itemize}
\end{frame}

%------------------------------------------------
\subsection{if}
%------------------------------------------------

%------------------------------------------------
\begin{frame}
\frametitle{while Loop}
\begin{itemize}
	\footnotesize
	\ttfamily
	\begin{block}{}
        \item[] if [ \textquotedbl\$(ls | wc -l)\textquotedbl -gt 5 ]; then
        \item[] 	echo youve got a lot of files in here
        \item[] else
        \item[]     echo do more work
        \item[] fi
	\end{block}	
	\begin{block}{}
        \item[] if [ \textquotedbl\$(ls | wc -l)\textquotedbl -gt 5 ]; then echo youve got a lot of files in here; else echo do more work; fi
	\end{block}
	\sffamily
	\tiny
	\item[] Spacing is, as always, important
	\item[] When in doubt check your variables by echo'ing them
\end{itemize}
\end{frame}


%------------------------------------------------
\section{Data Example}
%------------------------------------------------

%------------------------------------------------
\begin{frame}
\frametitle{But... How does this help?}
\begin{itemize}
	\item<1-> So, how does this help solve the problem of downloading data?
	\begin{itemize}
		\item<5-> A loop can automate downloading many files
	\end{itemize}
	\item<2-> How can you use these tools to download a set of files?
	\begin{itemize}
		\item<6-> Use a \texttt{for} loop to iterate across SRR numbers
	\end{itemize}
	\item<3-> What other ways could this make data management easier?
	\item<4-> Think file naming...
\end{itemize}
\end{frame}

%------------------------------------------------
\section{Dotstorming}
%------------------------------------------------

%------------------------------------------------
\begin{frame}
\frametitle{Dotstorming Question}
\begin{itemize}
	\small
	\item<1-> I've been impressed with everyone's ability to digest class material, complete difficult assignments, and work together with your groups
	\item<2-> I'd like to make sure that the course stays relevant and helpful
	\item<3-> So with that in mind:  
	\item<3-> What task/part of your project are you most concerned about your ability to complete?
	\item<3-> \href{https://dotstorming.com/b/59d4d3ed03432ec1203b6c4b}{Dotstorming Board}:
	\item<3-> \href{https://dotstorming.com/b/59d4d3ed03432ec1203b6c4b}{https://dotstorming.com/b/59d4d3ed03432ec1203b6c4b}
\end{itemize}
\end{frame}

%------------------------------------------------
\begin{frame}
\frametitle{Batch Download Cluster Job}
\begin{itemize}
	\item An example of a for loop to use fastq-dump
	\ttfamily
	\scriptsize
	\begin{block}{}
		\item[] \#!/bin/bash
		\item[] \#
		\item[] \#SBATCH --job-name=srr\_dwnld
		\item[] \#SBATCH --output=/work/cc216/490S/cc216/srr\_dwnld.out
		\item[] \#SBATCH --error=/work/cc216/490S/cc216/srr\_dwnld.err
		\item[] \#SBATCH --mail-user=cc216@duke.edu
		\item[] \#SBATCH --mem=2G
		\item[] \#SBATCH --nodes=1
		\item[] 
		\item[] cd /work/cc216/490S/cc216/
		\item[] 
		\item[] for n in `cat SRR\_Acc\_List.txt'
		\item[] do
		\item[]		echo \$n
		\item[] 	fastq-dump -Z --split-files \$n
		\item[] done
	\end{block}
\end{itemize}
\end{frame}


%------------------------------------------------
\begin{frame}
\Huge{\centerline{The End}}
\end{frame}

%----------------------------------------------------------------------------------------

\end{document} 