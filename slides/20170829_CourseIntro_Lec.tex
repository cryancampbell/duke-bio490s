%%%%%%%%%%%%%%%%%%%%%%%%%%%%%%%%%%%%%%%%%
% Beamer Presentation
% LaTeX Template
% Version 1.0 (10/11/12)
%
% This template has been downloaded from:
% http://www.LaTeXTemplates.com
%
% License:
% CC BY-NC-SA 3.0 (http://creativecommons.org/licenses/by-nc-sa/3.0/)
%
%%%%%%%%%%%%%%%%%%%%%%%%%%%%%%%%%%%%%%%%%

%----------------------------------------------------------------------------------------
%	PACKAGES AND THEMES
%----------------------------------------------------------------------------------------

\documentclass{beamer}

\mode<presentation> {

% The Beamer class comes with a number of default slide themes
% which change the colors and layouts of slides. Below this is a list
% of all the themes, uncomment each in turn to see what they look like.

%\usetheme{default}
%\usetheme{AnnArbor}
%\usetheme{Antibes}
%\usetheme{Bergen}
%\usetheme{Berkeley}
%\usetheme{Berlin}
%\usetheme{Boadilla}
%\usetheme{CambridgeUS}
%\usetheme{Copenhagen}
%\usetheme{Darmstadt}
%\usetheme{Dresden}
%\usetheme{Frankfurt}
%\usetheme{Goettingen}
%\usetheme{Hannover}
%\usetheme{Ilmenau}
%\usetheme{JuanLesPins}
%\usetheme{Luebeck}
\usetheme{Madrid}
%\usetheme{Malmoe}
%\usetheme{Marburg}
%\usetheme{Montpellier}
%\usetheme{PaloAlto}
%\usetheme{Pittsburgh}
%\usetheme{Rochester}
%\usetheme{Singapore}
%\usetheme{Szeged}
%\usetheme{Warsaw}

% As well as themes, the Beamer class has a number of color themes
% for any slide theme. Uncomment each of these in turn to see how it
% changes the colors of your current slide theme.

%\usecolortheme{albatross}
%\usecolortheme{beaver}
%\usecolortheme{beetle}
%\usecolortheme{crane}
%\usecolortheme{dolphin}
%\usecolortheme{dove}
%\usecolortheme{fly}
%\usecolortheme{lily}
%\usecolortheme{orchid}
%\usecolortheme{rose}
%\usecolortheme{seagull}
%\usecolortheme{seahorse}
%\usecolortheme{whale}
%\usecolortheme{wolverine}

%\setbeamertemplate{footline} % To remove the footer line in all slides uncomment this line
%\setbeamertemplate{footline}[page number] % To replace the footer line in all slides with a simple slide count uncomment this line

%\setbeamertemplate{navigation symbols}{} % To remove the navigation symbols from the bottom of all slides uncomment this line
}

\usepackage{graphicx} % Allows including images
\usepackage{booktabs} % Allows the use of \toprule, \midrule and \bottomrule in tables
\usepackage{hyperref}

%----------------------------------------------------------------------------------------
%	TITLE PAGE
%----------------------------------------------------------------------------------------

\title[Course Intro]{Introduction to Methods in Computational Biology and Genomics} % The short title appears at the bottom of every slide, the full title is only on the title page

\author{C. Ryan Campbell} % Your name
\institute[Duke] % Your institution as it will appear on the bottom of every slide, may be shorthand to save space
{
Duke University \\ % Your institution for the title page
\medskip
\textit{c.ryan.campbell@duke.edu} % Your email address
}
\date{\today} % Date, can be changed to a custom date

\begin{document}

\begin{frame}
\titlepage % Print the title page as the first slide
\end{frame}

\begin{frame}
\frametitle{Overview} % Table of contents slide, comment this block out to remove it
\tableofcontents % Throughout your presentation, if you choose to use \section{} and \subsection{} commands, these will automatically be printed on this slide as an overview of your presentation
\end{frame}

%----------------------------------------------------------------------------------------
%	PRESENTATION SLIDES
%----------------------------------------------------------------------------------------

%------------------------------------------------
\section{Course} % Sections can be created in order to organize your presentation into discrete blocks, all sections and subsections are automatically printed in the table of contents as an overview of the talk
%------------------------------------------------

\subsection{Course Intro} % A subsection can be created just before a set of slides with a common theme to further break down your presentation into chunks

%------------------------------------------------
\begin{frame}
\frametitle{Today's Goals}
%what students should know/learn today
\begin{itemize}
\item Get familiarized with course format
\item Meet each other and myself
\item Understand course expectations
\item Know my teaching methods and reasons for offering this course
\end{itemize}


\end{frame}
%------------------------------------------------
\begin{frame}
\frametitle{Course Objectives}
%modified from syllabus
\begin{block}{Objective 1}<1->
Learning programming basics and best practices within a biology framework (i.e. basic building blocks for students to use biological software)
\end{block}

\begin{block}{Objective 2}<2->
Learning the statistics that underlie these tools and the rapidly evolving suite of measures used in "genomics"
\end{block}

\begin{block}{Objective 3}<3->
Combine 1 and 2 to test biological hypotheses with RNAseq data and present those results in an "IMRD" format (pronounced EM-rod)
\end{block}


\end{frame}
%------------------------------------------------
\begin{frame}
\frametitle{Course Syllabus}
%go over main points
%from syllabus
\end{frame}
%------------------------------------------------
\begin{frame}
\frametitle{Course Grade}
%largely project based, partial throughout semester
%randomly based on class activity
%pick allotment
\end{frame}
%------------------------------------------------
\begin{frame}
\frametitle{Course Policies}
\begin{itemize}
\item<1-> Group Work vs. Own Work
\item<2-> No Direct Attendence Policy
\item<3-> Interact and Contribute in Class
\end{itemize}

%group work v. own work

%no direct attendence policy (but i reserve the right to add one)
\end{frame}
%------------------------------------------------
\begin{frame}
\frametitle{Day-to-day Course}
\begin{enumerate}
\item<1-> Bring a laptop
\begin{itemize}
	\item<2-> Yes, Every Day
\end{itemize}
\item<3-> Tuesdays = Lecture
\item<4-> Thursdays = Lab
\end{enumerate}
%bring a laptop
%Tues/Thurs
\end{frame}
%------------------------------------------------
\begin{frame}
\frametitle{Project}
\end{frame}
%------------------------------------------------
\begin{frame}
\frametitle{Research Tools}
\begin{itemize}
	\item R
	\item kallisto
	\item github
\end{itemize}
%what tools will you learn
\end{frame}
%------------------------------------------------
\begin{frame}
\frametitle{Programming Languages}
\begin{itemize}
	\item R
	\item bin/bash
\end{itemize}
%what languages will we work in
\end{frame}
%------------------------------------------------

\subsection{Teaching Philosophy}
%------------------------------------------------
\begin{frame}
\frametitle{Teaching Philosophy}
\begin{itemize}
	\item Clear Goals
	\item Active Learning
	\item Student Driven
\end{itemize}

%clear goals
%active learning
%student driven
\end{frame}
%------------------------------------------------
\begin{frame}
\frametitle{Clear Goals}
\begin{itemize}
	\item Presented before each class
	\item Concepts or skills to focus on
	\item Call me on it if I forget them
\end{itemize}
%
\end{frame}
%------------------------------------------------
\begin{frame}
\frametitle{Active Learning}
\begin{itemize}
	\item Student Participation
	\item A different style than lecture-based courses (Flipped courses fall into this category)
	\item Natural fit for smaller class size, advanced material, and learning skills
\end{itemize}

%
\end{frame}
%------------------------------------------------
\begin{frame}
\frametitle{Student Driven}
\begin{itemize}
	\item Student Participation Required
	\item Work through examples in class and apply to your own question (project)
	\item Many skills need to be practiced, not taught
	%bargain - i will try to avoid boring step by step tutorials where you just copy and paste
	% but you have to actually do the activities and learn actively
\end{itemize}

%
\end{frame}
%------------------------------------------------



\section{In-Class Activity}
%------------------------------------------------

\begin{frame}
\frametitle{In-Class Activity}
	\begin{enumerate}
	\item Pair up *randomly*
	\item Fill out this \href{https://goo.gl/forms/fnCaNIcwQiNiawcv1}{Google form}
	\item Make a slide in the \href{https://docs.google.com/presentation/d/1WxVQJGrBO8NxQ2ZJwvZy2W61qAQYqN9V8a_xRUflU7w/edit?usp=sharing}{Google Slideshow} for your partner with their answers, including (at minimum):
		\begin{itemize}
		\item Name
		\item Picture
		\item `Three words' response
		\item Two answers from `'Whimsy'
		\item Feel free to expand (see \href{https://docs.google.com/presentation/d/1WxVQJGrBO8NxQ2ZJwvZy2W61qAQYqN9V8a_xRUflU7w/edit?usp=sharing}{my slide} for reference)
		\end{itemize}
	\item Introduce partner to class
	\end{enumerate}
\end{frame}

%------------------------------------------------
\begin{frame}
  There are three important points:
  \begin{enumerate}
  \item<1-> A first one,
  \item<2-> a second one with a bunch of subpoints,
    \begin{itemize}
    \item first subpoint. (Only shown from second slide on!).
    \item<3-> second subpoint added on third slide.
    \item<4-> third subpoint added on fourth slide.
    \end{itemize}
  \item<5-> and a third one.
  \end{enumerate}
\end{frame}


%------------------------------------------------
\section{Computing and Genomics}
\subsection{Computer Requirements}
%------------------------------------------------
\begin{frame}
\frametitle{Computer Requirements}
%bring yours to every class
%run software from your computer
\end{frame}
%------------------------------------------------
\begin{frame}
\frametitle{Software}
%R-Studio
%github
%analysis software
\end{frame}
%------------------------------------------------
\begin{frame}
\frametitle{R-Studio}
%
\end{frame}
%------------------------------------------------
\begin{frame}
\frametitle{github}
%
\end{frame}
%------------------------------------------------
\begin{frame}
\frametitle{Analysis Software}
%fastqc
%trimmomatic
%kallisto
\end{frame}
%------------------------------------------------
\begin{frame}
\frametitle{Cluster Computing}
%
\end{frame}
%------------------------------------------------
\begin{frame}
\frametitle{Computer Languages}
%
\end{frame}
%------------------------------------------------





%------------------------------------------------
\subsection{Genomics: Why We're Here}
%------------------------------------------------
\begin{frame}
\frametitle{Hand Raising Request}
%Student Driven
%
\end{frame}
%------------------------------------------------
\begin{frame}
\frametitle{Genomics}
%what is genomics (class responses)
\end{frame}
%------------------------------------------------
\begin{frame}
\frametitle{Brief History of Sequencing}
%
\end{frame}
%------------------------------------------------
\begin{frame}
\frametitle{How Did We Get Here?}
%When was the first "sequencing"
\end{frame}
%------------------------------------------------
\begin{frame}
\frametitle{Last Generation Sequencing}
%Sanger & Low Output
\end{frame}
%------------------------------------------------
\begin{frame}
\frametitle{Next Generation Sequencing}
%What all it entails
\end{frame}
%------------------------------------------------
\begin{frame}
\frametitle{Sequencing Output}
%The plot that is in every talk
\end{frame}
%------------------------------------------------
\begin{frame}
\frametitle{Beyond NGS}
%many many many different directions
\end{frame}
%------------------------------------------------
\begin{frame}
\frametitle{Course Motivations}
%computational need
%lack of experience at this level
%cheap data
\end{frame}
%------------------------------------------------
\begin{frame}
\frametitle{Computational Need}
%important for biologists to be programmers
\end{frame}
%------------------------------------------------
\begin{frame}
\frametitle{Lack of Offerings}
%no courses teach NGS
\end{frame}
%------------------------------------------------
\begin{frame}
\frametitle{Cheap/Free Data}
%how much? ncbi/sra chart
\end{frame}
%------------------------------------------------
\begin{frame}
\frametitle{Personal Experience}
%right in my wheelhouse
\end{frame}
%------------------------------------------------

\begin{frame}
\Huge{\centerline{The End}}
\end{frame}

%----------------------------------------------------------------------------------------

\end{document} 